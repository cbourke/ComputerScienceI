\documentclass[12pt]{scrartcl}


\usepackage{epsfig,amssymb}

\usepackage{xcolor}
\usepackage{graphicx}
\usepackage{epstopdf}
\usepackage{multirow}

\definecolor{darkred}{rgb}{0.5,0,0}
\definecolor{darkgreen}{rgb}{0,0.5,0}
\usepackage{hyperref}
\hypersetup{
  letterpaper,
  colorlinks,
  linkcolor=red,
  citecolor=darkgreen,
  menucolor=darkred,
  urlcolor=blue,
  bookmarks=true,
  pdfpagemode=none,
  pdftitle={BYOD: Bring Your Own Device},
  pdflang={en},
  pdfauthor={Christopher M. Bourke},
  pdfcreator={$ $Id: cv-us.tex,v 1.28 2009/01/01 00:00:00 cbourke Exp $ $},
  pdfsubject={PhD Thesis},
  pdfkeywords={}
}

\usepackage{fullpage}
\usepackage{tikz}
\pagestyle{empty} %
\usepackage{subfigure}

\definecolor{MyDarkBlue}{rgb}{0,0.08,0.45}
\definecolor{MyDarkRed}{rgb}{0.45,0.08,0}
\definecolor{MyDarkGreen}{rgb}{0.08,0.45,0.08}

\definecolor{mintedBackground}{rgb}{0.95,0.95,0.95}
\definecolor{mintedInlineBackground}{rgb}{.90,.90,1}

\usepackage[newfloat=true]{minted}

\setminted{mathescape,
           linenos,
           autogobble,
           frame=none,
           framesep=2mm,
           framerule=0.4pt,
           %label=foo,
           xleftmargin=2em,
           xrightmargin=0em,
           %startinline=true,  %PHP only, allow it to omit the PHP Tags *** with this option, variables using dollar sign in comments are treated as latex math
           numbersep=10pt, %gap between line numbers and start of line
           style=default} %syntax highlighting style, default is "default"

\setmintedinline{bgcolor={mintedBackground}}
%doesn't work with the above workaround:
\setminted{bgcolor={mintedBackground}}
\setminted[text]{bgcolor={mintedBackground},linenos=false,autogobble,xleftmargin=1em}
%\setminted[php]{bgcolor=mintedBackgroundPHP} %startinline=True}
\SetupFloatingEnvironment{listing}{name=Code Sample}
\SetupFloatingEnvironment{listing}{listname=List of Code Samples}

\setlength{\parindent}{0pt} %
\setlength{\parskip}{.25cm}
\newcommand{\comment}[1]{}

\usepackage{amsmath}
\usepackage{algorithm2e}
\SetKwInOut{Input}{input}
\SetKwInOut{Output}{output}
%NOTE: you can embed algorithms in solutions, but they cannot be floating objects; use [H] to make them non-floats

\usepackage{lastpage}

%\usepackage{titling}
\usepackage{fancyhdr}
\renewcommand*{\titlepagestyle}{fancy}
\pagestyle{fancy}
%\renewcommand*{\titlepagestyle}{fancy}
%\fancyhf{}
\rhead{~}
\lhead{~}
\renewcommand{\headrulewidth}{0.0pt}
\renewcommand{\footrulewidth}{0.4pt}
\lfoot{\Title\ -- Computer Science I}
\cfoot{~}
\rfoot{\thepage\ / \pageref*{LastPage}}

\makeatletter
\title{Tips for Succeeding in This and Other Online Courses}\let\Title\@title
\subtitle{Computer Science I\\
{\small
\vskip.5cm
Department of Computer Science \& Engineering \\
University of Nebraska--Lincoln}
\vskip-2cm}
%\author{Dr.\ Chris Bourke}
\date{~}
\makeatother

\begin{document}

\maketitle

\hrule

\section{Treat It Like a Regular Course}

This may be an online course, but it is every bit as rigorous and challenging
as the traditional face-to-face version and you need to approach it as such.
In fact, online courses are actually \emph{more} challenging to take than 
a traditional course because you have to have the self-discipline to overcome
the additional challenges that an online course provides.  Without the expectation
and structure of showing up to a physical classroom, you can get complacent, 
feel detached and your motivation may suffer as a result.

Hold yourself accountable and make the commitment that you \emph{will} succeed
in this course and that you \emph{will} take the necessary steps to doing so.

\section{Get and Stay Organized}

\begin{itemize}
  \item Set a plan and set a regular schedule on when and how you will
  engage with the material in this course. For example, though you may be able to 
  watch the lectures and videos or do the readings asynchronously, you
  should establish a routine and watch/read at the same time every day.
  \item Schedule regular times to work on class assignments and build in
  slack time: add more time to your estimate on how much time you'll need 
  to complete it.  It is never wrong to get done early and the slack time
  will allow you to deal with things that may come up or to seek additional
  help if you get stuck.
  \item Work \emph{everyday}: programming (and Computer Science in general) is
  something you only get good at with practice.  It requires time and
  hands-on effort which is best achieved with daily practice.  Like any skill, 
  it can atrophy if not practiced.  Don't get into the ``cram'' mentality 
  and think that you can pack in 5 hours of material in one sitting.  This
  stuff takes time, spread that 5 hours over 5 days, one hour per day.
  \item Organize your work and keep a calendar with reminders.  All assignments
  are posted in Canvas and you can export due dates to iCal or ICS formats.
  \item Organize your work my maintaining a TODO list or use an organization
  app.
  \item Schedule time for breaks, look into the \href{https://en.wikipedia.org/wiki/Pomodoro_Technique}{Pomodoro Technique}
  \item Be sure to include \emph{all} aspects of your life including recreation, 
  health, social engagements, etc. in your time management.  
  \item Periodically reassess your daily and weekly schedules and make appropriate
  adjustments.  
\end{itemize}  
  
\section{Create a Regular Study Space}

If you're not sitting in a class or lab it may not \emph{feel}
like a class which can easily lead to distractions.  This is
especially true if you have a TV, phone, games, etc.\ close by
presenting a temptation.  To combat this, establish a good working 
area away from distractions.  Don't take your work outside this
area nor your recreation into this area.  Keep your worlds separate.

\section{Network, Connect, and Stay Engaged}

It can be difficult to not see other students or your instructor face-to-face
which can leave you with a sense of disconnectedness which can lead to feelings
of isolation and a negative effect on motivation.  Combat this by actively 
staying connected to the course, your instructor, Learning Assistants and peer
students by:

\begin{itemize}
  \item Regularly reading and participating in the online discussion forum (Piazza)
  \item Regularly attending virtual office hours via Zoom (even if it is just
  to ``hang out'' while you work).  
  \item Reach out to classmates to find a study partner or collaborator and keep
  in touch with collaboration tools (Zoom, Repl.it, etc.)
\end{itemize}

\section{Utilize Resources}

\begin{itemize}
  \item Familiarize yourself with all the resources available in the course and 
  \emph{use} them!
  \item Don't wait to ask for help when you need it.
  \item Use Piazza, our online discussion board: there is no such thing as a
  stupid question.  Even so, you can avoid embarrassment by asking anonymously!
\end{itemize}


\section{Stay Healthy}

This advice goes beyond just this course or any course for that matter.  If
good habits apply to a course, then they can be applied to everything in your
life.  

\begin{itemize}
  \item Get into a good routine: wake up at the same time every day (and weekend)
  and go to bed at the same time.   Get enough sleep.
  \item Exercise; take a walk, bike, go to the gym, etc.
  \item Engage in a healthy diet, physical health is the number one factor to
  mental health
  \item Don't cut recreation (video games, netflix, etc.) but regulate it.  
  Schedule time for it but stick to that time.  
  \item Set a balance: sleep more than you study (or work); study more than you party
  and party as much as you can.
\end{itemize}

\section{Expect to be Challenged}
It's OK to struggle. College is supposed to be hard. Very few students 
sail through without struggling at some point.

Don't compare yourself to others, compare yourself to yourself and focus
on your own improvement throughout the semester and beyond.

\end{document}
