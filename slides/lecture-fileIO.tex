%%%%%%%%%%%%%%%%%%%%%%%%%%%%%%%%%%%%%%%%%%%%%%%%%%%%%
%
%  Template
%  Beamer Presentation by Chris Bourke
%
%%%%%%%%%%%%%%%%%%%%%%%%%%%%%%%%%%%%%%%%%%%%%%%%%%%%%%%%%%%%%%%%%%%%%%%

\documentclass[]{beamer}
%\documentclass[handout]{beamer}

\geometry{papersize={16cm,9cm}}

% For handout version:
%\usetheme[hideothersubsections,slidenumbers]{UNLTheme}
\usetheme[hideothersubsections]{UNLTheme}
\usepackage{amssymb}
\input{StandardCommands}
\usepackage[linesnumbered,ruled,vlined]{algorithm2e}
\SetKwComment{Comment}{//}{}
\DontPrintSemicolon
\SetKwSty{textsc} %
%\SetAlFnt{\scriptsize} %
\SetKwInOut{Input}{Input} %
\SetKwInOut{Output}{Output} %
%\setalcapskip{1em} % changed to
\SetAlCapSkip{1em}
\setlength{\algomargin}{2em} %
%\Setvlineskip{0em} % changed to:
\SetVlineSkip{0em}

\usepackage{tikz}
\usetikzlibrary{fadings}
\usetikzlibrary{shapes.geometric,shapes.symbols}
\usetikzlibrary{calc,shapes.multipart,chains,arrows}
\usetikzlibrary{arrows.meta,calc,shapes.multipart,chains,arrows}
%\usetikzlibrary{calc,shapes.multipart,chains,arrows}
%%\usetikzlibrary{backgrounds}
\usetikzlibrary{backgrounds}
\usetikzlibrary{decorations.pathreplacing}
\usetikzlibrary{decorations.pathmorphing}
\tikzset{onslide/.code args={<#1>#2}{%
  \only<#1>{\pgfkeysalso{#2}} % \pgfkeysalso doesn't change the path
}}
\tikzset{temporal/.code args={<#1>#2#3#4}{%
  \temporal<#1>{\pgfkeysalso{#2}}{\pgfkeysalso{#3}}{\pgfkeysalso{#4}} % \pgfkeysalso doesn't change the path
}}


\tikzset{
    fading speed/.code={
        \pgfmathtruncatemacro\tikz@startshading{50-(100-#1)*0.25}
        \pgfmathtruncatemacro\tikz@endshading{50+(100-#1)*0.25}
        \pgfdeclareverticalshading[%
            tikz@axis@top,tikz@axis@middle,tikz@axis@bottom%
        ]{axis#1}{100bp}{%
            color(0bp)=(tikz@axis@bottom);
            color(\tikz@startshading)=(tikz@axis@bottom);
            color(50bp)=(tikz@axis@middle);
            color(\tikz@endshading)=(tikz@axis@top);
            color(100bp)=(tikz@axis@top)
        }
        \tikzset{shading=axis#1}
    }
}

\usepackage{multirow}
\usepackage{multicol}

\definecolor{steelblue}{rgb}{0.2745,0.5098,0.7059}
\definecolor{limegreen}{RGB}{50,205,50}
\hypersetup{
    colorlinks = true,
    urlcolor = {steelblue},
    linkbordercolor = {white}
}

\definecolor{mintedBackground}{rgb}{0.95,0.95,0.95}
\definecolor{mintedInlineBackground}{rgb}{.90,.90,1}

%\usepackage{newfloat}
\usepackage{minted}
\setminted{mathescape,
               linenos,
               autogobble,
               frame=none,
               fontsize=\small,
               framesep=2mm,
               framerule=0.4pt,
               %label=foo,
               xleftmargin=2em,
               xrightmargin=0em,
               startinline=true,  %PHP only, allow it to omit the PHP Tags *** with this option, variables using dollar sign in comments are treated as latex math
               numbersep=10pt, %gap between line numbers and start of line
               style=default, %syntax highlighting style, default is "default"
               			    %gallery: http://help.farbox.com/pygments.html
			    	    %list available: pygmentize -L styles
               bgcolor=mintedBackground} %prevents breaking across pages
               
\setmintedinline{bgcolor={mintedInlineBackground}}
\setminted[text]{bgcolor={},linenos=false,autogobble,xleftmargin=1em}

\tikzstyle{decision} = [diamond, draw, fill=yellow!20, 
    text width=6em, text badly centered, node distance=5cm, inner sep=0pt]
\tikzstyle{block} = [rectangle, draw, fill=blue!20, 
    text width=5em, text centered, node distance=5cm, minimum height=4em]
\tikzstyle{action} = [rectangle, draw, fill=green!20, 
    text width=5em, text centered, rounded corners, node distance=5cm, minimum height=4em]
\tikzstyle{line} = [draw, -latex']

\title[~]{Computer Science I}
\subtitle{File Input/Output}
\author[~]{Dr.\ Chris Bourke\\ \email{cbourke@cse.unl.edu}} %
\date{~}

\begin{document}

\begin{frame}
  \titlepage
\end{frame}

\setbeamertemplate{section in toc}{\inserttocsectionnumber.~\inserttocsection}
\setbeamercolor{section in toc}{fg=black}
%\setbeamertemplate{subsection in toc}{~} %\inserttocsubsection\\}

\begin{frame}
  \frametitle{Outline}
%{\footnotesize
%\begin{NoHyper}
%  \tableofcontents[hideallsubsections]
%\end{NoHyper}
%}

\setbeamercolor{enumerate item}{bg=white,fg=black}
\setbeamercolor{item}{bg=white,fg=black}
\setbeamercolor{item projected}{bg=white,fg=black}
\setbeamercolor{enumerate subitem}{fg=red!80!black}
\setbeamertemplate{enumerate items}[default]
\begin{enumerate}
  \item Introduction \& File Output
  \item File Input
  \item Binary Files
  \item Exercises
\end{enumerate}

\end{frame}

\section{Introduction}

\begin{frame}
    \frametitle{}
    \framesubtitle{}
    
    \begin{center}
    {\Huge Part I: Introduction \& File Output}\\
    {\Large ~}
    \end{center}

\end{frame}

\begin{frame}[fragile]
  \frametitle{Overview}
  \framesubtitle{}

\begin{itemize}[<+->]
  \item A \emph{file} is a unit of stored memory, usually on disk
  \item The following are also files:
  \begin{itemize}
    \item Directories
    \item Buffers (standard input/output)
    \item Programs, stored and running
    \item Network sockets
  \end{itemize}
  \item Files may be plaintext (ASCII) or binary
  \item Or: plaintext data not intended for human consumption
  \item CSV, XML, JSON, base-64 encoding
\end{itemize}

\end{frame}

\begin{frame}[fragile]
  \frametitle{Overview}
  \framesubtitle{}

\begin{itemize}[<+->]

  \item You read from a file (input) or write to a file (output)
  \item Three basic steps for file I/O:
  \begin{itemize}
    \item Open the file
    \item Process the file
    \item Close the file
  \end{itemize}
\end{itemize}
%buffered vs non-buffered?

\end{frame}

\subsection{Opening Files}

\begin{frame}[fragile]
  \frametitle{Opening Files}
  \framesubtitle{}

\begin{itemize}[<+->]
  \item Files are supported in C using a \emph{file pointer}
  \item \mintinline{c}{FILE *} points to a location in a file
  \item The \mintinline{c}{fopen()} function opens a file and returns a file pointer
  \item Initially: points to the start of the file
  \item As you read through the file, the pointer is updated
  \item Returns \mintinline{c}{NULL} if unsuccessful
  \item Also called a \emph{handle}
\end{itemize}

\end{frame}

\begin{frame}[fragile]
  \frametitle{Opening Files}
  \framesubtitle{}

\begin{itemize}[<+->]
  \item \mintinline{c}{FILE * fopen(const char *filename, const char *mode);} 
  \item First argument: path and name of the file to open
  \item Second argument: \emph{mode} to open it up in
  \item File input: \mintinline{c}{"r"} for \emph{reading}
  \item File output: \mintinline{c}{"w"} for \emph{writing}
  \item Demonstration  
\end{itemize}

\end{frame}

\begin{frame}[fragile]
  \frametitle{Demonstration}
  \framesubtitle{}

\begin{minted}{c}
//open a file data.txt in the current directory for reading:
FILE *f = fopen("data.txt", "r");

//open a file data.txt for writing:
FILE *f = fopen("data.txt", "w");

//you can also use relative paths
FILE *f = fopen("../../data.txt", "r");

//absolute path:
FILE *f = fopen("/etc/shadow", "r");

//error checking: 
if(f == NULL) {
  printf("Unable to open file!\n");
}
\end{minted}

\end{frame}

\begin{frame}[fragile]
  \frametitle{Pitfalls}
  \framesubtitle{}

\begin{itemize}[<+->]
  \item Your program must have proper permissions to read/write a file
  \item Opening a file for writing will create it if it does not already exist
  \item Opening an existing file for writing will \emph{clobber} it
  \item Failure to \emph{close} a file \emph{may} lead to data corruption
  \item Closing a file:\\
  \mintinline{c}{fclose(f);}
\end{itemize}

\end{frame}

\subsection{Paths}

\begin{frame}[fragile]
  \frametitle{Paths}
  \framesubtitle{}

\begin{itemize}[<+->]
  \item Current Working Directory: \mintinline{text}{.}
  \item File System Root: \mintinline{text}{/}
  \item One directory ``up'' the hierarchy \mintinline{text}{..}
  %\item Directory Delimiter: \mintinline{text}{/}
\end{itemize}

\end{frame}

\subsection{File Output}

\begin{frame}[fragile]
  \frametitle{File Output}
  \framesubtitle{}

\begin{itemize}[<+->]
  \item Many ways to output to a file
  \item Easiest and most simple: \mintinline{c}{fprintf()}
  \item Same functionality as \mintinline{c}{printf()} and \mintinline{c}{sprintf()}
  \item Takes a \mintinline{c}{FILE*} as its first argument and prints to it
  \item Demonstration
\end{itemize}

\end{frame}

\begin{frame}[fragile]
  \frametitle{File Output}
  \framesubtitle{}

\begin{minted}{c}
int a = 42;

FILE *f = fopen("data.txt", "w");

fprintf(f, "Hello World!\n");
fprintf(f, "a = %d\n");
fprintf(f, "pi is %.4f\n", M_PI);

fclose(f);
\end{minted}

\end{frame}


\section{File Input}

\begin{frame}
  \frametitle{}
  \framesubtitle{}
    
    \begin{center}
    {\Huge Part II: File Input}\\
    {\Large ~}
    \end{center}

\end{frame}

\begin{frame}[fragile]
  \frametitle{File Input}
  \framesubtitle{}

\begin{itemize}[<+->]
  \item There are many (dangerous) ways of reading from a file
  \item Best to limit the amount of data read so it is predictable
  \item Avoid ``buffer overflows'' (strings where we store the data)
  \item Focus on two useful functions:
  \item \mintinline{c}{fgetc()} gets a single character from a file
  \item \mintinline{c}{fgets()} gets (up to) an entire line from a file 
\end{itemize}  
  
\end{frame}

\begin{frame}[fragile]
  \frametitle{fgetc}
  \framesubtitle{}

\begin{itemize}[<+->]
  \item \mintinline{c}{int fgetc(FILE *f);}
  \item Reads a single \mintinline{c}{char} from the file \mintinline{c}{f}
  \item Returns the ASCII value of the character
  \item Automatically advances the file pointer to the next character
  \item Returns a special flag, \mintinline{c}{EOF} when it gets to the end-of-file
  \item Demonstration
\end{itemize}
  
\end{frame}

\begin{frame}[fragile]
  \frametitle{fgetc}
  \framesubtitle{}

\begin{minted}{c}
#include<stdlib.h>
#include<stdio.h>
#include<string.h>

int main(int argc, char **argv) {

  FILE *f = fopen("./data/students.csv", "r");
  char c = fgetc(f);
  while(c != EOF) {
    printf("c = %c\n", c);
    c = fgetc(f);
  }
  fclose(f);

  return 0;
}
\end{minted}
  
\end{frame}


\begin{frame}[fragile]
  \frametitle{fgets}
  \framesubtitle{}

\begin{itemize}[<+->]
  \item \mintinline{c}{char * fgets(char *str, int size, FILE *f);}
  \item Reads at most \mintinline{c}{size-1} characters from \mintinline{c}{f}
  \item Places the result into \mintinline{c}{str} (a ``buffer'')
  \item Automatically null-terminates the string
  \item Stops early if the end-of-line character \mintinline{text}{\n} is encountered
  \item \emph{Retains} the end-of-line character in \mintinline{c}{str}
  \item Returns \mintinline{c}{NULL} when it reaches the end of file
  \item Alternatively (for both): \mintinline{c}{int feof(FILE *f);}
  \item Demonstration
\end{itemize}

\end{frame}

\begin{frame}[fragile]
  \frametitle{fgets}
  \framesubtitle{}

\begin{minted}[fontsize=\tiny]{c}
#include<stdlib.h>
#include<stdio.h>
#include<string.h>

int main(int argc, char **argv) {

  FILE *f = fopen("./data/students.csv", "r");

  char buffer[100];
  char *line = fgets(buffer, 100, f);
  while(line != NULL) {
    //chomp out the endline character:
    int n = strlen(buffer);
    if(buffer[n-1] == '\n') {
      buffer[n-1] = '\0';
    }
    printf("line = %s\n", line);
    line = fgets(buffer, 100, f);
  }
  fclose(f);
  return 0;
}
\end{minted}
  
\end{frame}


\section{Binary Files}

\begin{frame}
  \frametitle{}
  \framesubtitle{}
    
    \begin{center}
    {\Huge Part III: Binary Files}\\
    {\Large ~}
    \end{center}

\end{frame}

\begin{frame}[fragile]
  \frametitle{Binary Files}
  \framesubtitle{}

\begin{itemize}[<+->]
  \item Most data formats are \emph{binary}: raw bits and bytes
  \item May have specific format and magic number identifiers
  \item GIF format: \url{https://en.wikipedia.org/wiki/GIF}
  \item Often more efficient: less space, easier to read/write
  \item C: \mintinline{c}{fread} (reading) and \mintinline{c}{fwrite} (writing) binary data
  \item Reads/writes multiple pieces of data at once
\end{itemize}
  
\end{frame}

\begin{frame}[fragile]
  \frametitle{Binary Files}
  \framesubtitle{}
  
\begin{itemize}[<+->]
    \item \mintinline{c}{size_t fread(void *ptr, size_t size, size_t n, FILE *f);}
    \item \mintinline{c}{size_t fwrite(const void *ptr, size_t size, size_t n, FILE *f);}
    \item Both take the same arguments
    \begin{itemize}
      \item \mintinline{c}{ptr} -- pointer to the data to be read/written
      \item \mintinline{c}{size} -- number of bytes for each item (use \mintinline{c}{sizeof})
      \item \mintinline{c}{n} -- number of items to be read/written
      \item \mintinline{c}{f} -- file pointer
    \end{itemize}
    \item Demonstration
\end{itemize}

\end{frame}

\section{Exercises}

\begin{frame}
  \frametitle{}
  \framesubtitle{}
    
    \begin{center}
    {\Huge Part IV: Exercises}\\
    {\Large ~}
    \end{center}

\end{frame}

\begin{frame}[fragile]
  \frametitle{File I/O}
  \framesubtitle{Exercise}

Write a program to process a CSV file with student information including
last name, first name, NUID, and GPA to prepare a Dean's list.  Output to
a separate file all student names whose GPA is greater than 3.5, but only include
their first name and last name (one to a line).

\end{frame}

\begin{frame}[fragile]
  \frametitle{File I/O}
  \framesubtitle{Exercise}

\begin{itemize}[<+->]
  \item Passwords are stored using cryptographic hash functions
  \item   {\footnotesize
$hash(\textrm{password}) \rightarrow \texttt{0x5e884898da28047151d0e56f8dc6292773603d0d6aabbdd62a11ef721d1542d8}$}
  \item Common for users to use dictionary words as passwords
  \item They can easily be broken using a \emph{dictionary attack}
  \item Exercise: dictionary attack a SHA-256 hashed password:
  {\footnotesize
  \texttt{0xaa97302150fce811425cd84537028a5afbe37e3f1362ad45a51d467e17afdc9c}
  }
\end{itemize}
  
\end{frame}


\end{document} 
