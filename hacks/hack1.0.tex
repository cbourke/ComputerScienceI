\documentclass[12pt]{scrartcl}


\usepackage{epsfig,amssymb}

\usepackage{xcolor}
\usepackage{graphicx}
\usepackage{epstopdf}
\usepackage{multirow}
\usepackage{float}

\definecolor{darkred}{rgb}{0.5,0,0}
\definecolor{darkgreen}{rgb}{0,0.5,0}
\usepackage[pdfusetitle]{hyperref}
\hypersetup{
  letterpaper,
  colorlinks,
  linkcolor=red,
  citecolor=darkgreen,
  menucolor=darkred,
  urlcolor=blue,
  pdfpagemode=none,
}

\usepackage{fullpage}
\usepackage{tikz}
\pagestyle{empty} %
%obsolete: \usepackage{subfigure}
%use: 
\usepackage{subcaption}

\definecolor{MyDarkBlue}{rgb}{0,0.08,0.45}
\definecolor{MyDarkRed}{rgb}{0.45,0.08,0}
\definecolor{MyDarkGreen}{rgb}{0.08,0.45,0.08}

\definecolor{mintedBackground}{rgb}{0.95,0.95,0.95}
\definecolor{mintedInlineBackground}{rgb}{.90,.90,1}

\usepackage[newfloat=true]{minted}

\setminted{mathescape,
           linenos,
           autogobble,
           frame=none,
           framesep=2mm,
           framerule=0.4pt,
           %label=foo,
           xleftmargin=2em,
           xrightmargin=0em,
           %startinline=true,  %PHP only, allow it to omit the PHP Tags *** with this option, variables using dollar sign in comments are treated as latex math
           numbersep=10pt, %gap between line numbers and start of line
           style=default} %syntax highlighting style, default is "default"

\setmintedinline{bgcolor={mintedBackground}}
%doesn't work with the above workaround:
\setminted{bgcolor={mintedBackground}}
\setminted[text]{bgcolor={mintedBackground},linenos=false,autogobble,xleftmargin=1em}
%\setminted[php]{bgcolor=mintedBackgroundPHP} %startinline=True}
\SetupFloatingEnvironment{listing}{name=Code Sample}
\SetupFloatingEnvironment{listing}{listname=List of Code Samples}

\setlength{\parindent}{0pt} %
\setlength{\parskip}{.25cm}
\newcommand{\comment}[1]{}

\usepackage{amsmath}
\usepackage{algorithm2e}
\SetKwInOut{Input}{input}
\SetKwInOut{Output}{output}
%NOTE: you can embed algorithms in solutions, but they cannot be floating objects; use [H] to make them non-floats

\usepackage{lastpage}

%\usepackage{titling}
\usepackage{fancyhdr}
\renewcommand*{\titlepagestyle}{fancy}
\pagestyle{fancy}
%\fancyhf{}
%\rhead{Computer Science I}
%\lhead{Guides and tutorials}
\renewcommand{\headrulewidth}{0.0pt}
\renewcommand{\footrulewidth}{0.4pt}
\lfoot{\Title\ -- Computer Science I}
\cfoot{~}
\rfoot{\thepage\ / \pageref*{LastPage}}


\makeatletter
\title{Hack 1.0}\let\Title\@title
\subtitle{Computer Science I\\
{\small
\vskip1cm
Department of Computer Science \& Engineering \\
University of Nebraska--Lincoln}
\vskip-1cm}
%\author{Dr.\ Chris Bourke}
\date{~}
\makeatother

\begin{document}

\maketitle

\hrule

\section*{Introduction}

Hack session activities are small weekly programming assignments intended
to get you started on full programming assignments.  Collaboration is allowed
and, in fact, \emph{highly encouraged}.  You may start on the activity before
your hack session, but during the hack session you must either be actively 
working on this activity or \emph{helping others} work on the activity.
You are graded using the same rubric as assignments so documentation, style, 
design and correctness are all important.  This activity is \textbf{due 
at 23:59:59 on the Friday} in the week in which it is assigned according 
to the CSE system clock.

%TODO: rubric 
% instructions
% documentation
% style
% design including input validation
% correctness

\section*{Problem Statement}

This activity will help you get started using git, a distributed 
version control system.  In particular, you'll be using GitHub, 
\url{https://github.com/} as your git system.  Before you begin, 
you will need to create an account on GitHub and read the course 
tutorial on getting started with git: 

\url{http://cse.unl.edu/~cbourke/gitTutorial.pdf}.  

Alternatively, there are several resources provided below.  We
recommend that you install and use the GitHub Desktop Client for
this course: \url{https://desktop.github.com/}.

To get started using Git/GitHub, you will create a basic 
repository from scratch, push it to GitHub and then collaborate 
with at least one other partner to share changes.
  
To keep things simple, your project will be a simple ``Hello World'' 
style program but instead of printing ``Hello World'', it will 
print your name and major. Once you have created this simple project, 
push it to GitHub.  Then have your partner clone it and add their own 
name to the program (so that it prints both of your names).  Then
add them as a collaborator so they can push their changes to your
repository.  Alternatively, you can have them make a pull request 
and pull their changes to your repository.  You will both need to 
repeat this exercise so that you both have repositories in your 
GitHub profile.

You are \emph{highly encouraged} to start using git/GitHub (or 
something similar) for all of your future assignments 
but be sure to commit code to a \emph{private} repository 
so that you do not violate the department's academic integrity 
policy.  For this exercise, however, keep your repositories 
public so that we can grade you (you are
being graded on the process, not the code itself). 

\section*{Instructions}

\begin{itemize}
  \item You are encouraged to collaborate any number of students 
  before, during, and after your scheduled hack session.    
  \item Put your GitHub URL into a plain text file named 
  \mintinline{text}{readme.md}.  Include the name(s) of everyone 
  who worked together on this activity as well.  Turn this file
  in using webhandin.  Each individual student will need
  to hand in their own copy and will receive their own individual
  grade.
\end{itemize}
  
\section*{Additional Resources}

\begin{itemize} 
  \item GitHub Desktop Client: \url{https://desktop.github.com/} 
  \item Video tutorial on Github Desktop: \url{https://www.youtube.com/watch?v=kFix7UDJ7LA}
  \item Interactive git tutorial: \url{https://try.github.io/levels/1/challenges/1}
  \item Pro Git, free online book: \url{https://git-scm.com/book/en/v2}
  \item Documentation on GIthub Desktop: \url{https://help.github.com/desktop/guides/getting-started-with-github-desktop/}
\end{itemize}


  


\end{document}
