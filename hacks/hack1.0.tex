\documentclass[12pt]{scrartcl}


\usepackage{epsfig,amssymb}

\usepackage{xcolor}
\usepackage{graphicx}
\usepackage{epstopdf}
\usepackage{multirow}
\usepackage{float}

\definecolor{darkred}{rgb}{0.5,0,0}
\definecolor{darkgreen}{rgb}{0,0.5,0}
\usepackage[pdfusetitle]{hyperref}
\hypersetup{
  letterpaper,
  colorlinks,
  linkcolor=red,
  citecolor=darkgreen,
  menucolor=darkred,
  urlcolor=blue,
  pdfpagemode=none,
}

\usepackage{fullpage}
\usepackage{tikz}
\pagestyle{empty} %
%obsolete: \usepackage{subfigure}
%use: 
\usepackage{subcaption}

\definecolor{MyDarkBlue}{rgb}{0,0.08,0.45}
\definecolor{MyDarkRed}{rgb}{0.45,0.08,0}
\definecolor{MyDarkGreen}{rgb}{0.08,0.45,0.08}

\definecolor{mintedBackground}{rgb}{0.95,0.95,0.95}
\definecolor{mintedInlineBackground}{rgb}{.90,.90,1}

\usepackage[newfloat=true]{minted}

\setminted{mathescape,
           linenos,
           autogobble,
           frame=none,
           framesep=2mm,
           framerule=0.4pt,
           %label=foo,
           xleftmargin=2em,
           xrightmargin=0em,
           %startinline=true,  %PHP only, allow it to omit the PHP Tags *** with this option, variables using dollar sign in comments are treated as latex math
           numbersep=10pt, %gap between line numbers and start of line
           style=default} %syntax highlighting style, default is "default"

\setmintedinline{bgcolor={mintedBackground}}
%doesn't work with the above workaround:
\setminted{bgcolor={mintedBackground}}
\setminted[text]{bgcolor={mintedBackground},linenos=false,autogobble,xleftmargin=1em}
%\setminted[php]{bgcolor=mintedBackgroundPHP} %startinline=True}
\SetupFloatingEnvironment{listing}{name=Code Sample}
\SetupFloatingEnvironment{listing}{listname=List of Code Samples}

\setlength{\parindent}{0pt} %
\setlength{\parskip}{.25cm}
\newcommand{\comment}[1]{}

\usepackage{amsmath}
\usepackage{algorithm2e}
\SetKwInOut{Input}{input}
\SetKwInOut{Output}{output}
%NOTE: you can embed algorithms in solutions, but they cannot be floating objects; use [H] to make them non-floats

\usepackage{lastpage}

%\usepackage{titling}
\usepackage{fancyhdr}
\renewcommand*{\titlepagestyle}{fancy}
\pagestyle{fancy}
%\fancyhf{}
%\rhead{Computer Science I}
%\lhead{Guides and tutorials}
\renewcommand{\headrulewidth}{0.0pt}
\renewcommand{\footrulewidth}{0.4pt}
\lfoot{\Title\ -- Computer Science I}
\cfoot{~}
\rfoot{\thepage\ / \pageref*{LastPage}}

%remove headers
\rhead{~}
\lhead{~}


\makeatletter
\title{Hack 1.0}\let\Title\@title
\subtitle{Computer Science I\\
{\small
\vskip1cm
Department of Computer Science \& Engineering \\
University of Nebraska--Lincoln}
\vskip-1cm}
%\author{Dr.\ Chris Bourke}
\date{~}
\makeatother

\begin{document}

\maketitle

\hrule

\section*{Introduction}

Hack session activities are small weekly programming assignments intended
to get you started on full programming assignments.  Collaboration is allowed
and, in fact, \emph{highly encouraged}.  You may start on the activity before
your hack session, but during the hack session you must either be actively 
working on this activity or \emph{helping others} work on the activity.
You are graded using the same rubric as assignments so documentation, style, 
design and correctness are all important.  This activity is due at 23:59:59
on the Monday following the hack session in which it is assigned according
to the CSE system clock.

%TODO: rubric 
% instructions
% documentation
% style
% design including input validation
% correctness



For correctness:
\begin{itemize}
  \item Code itself needs to be correct: 4 pts
  \item There should be more than one commit: 4 pts
  \item All commits should have a descriptive comment: 3 pts
  \item There must be at least 2 contributors: 5 pts
\end{itemize}

\section*{Problem Statement}

An essential tool when developing software is a \emph{version control system}
(VCS).  As you develop software you will make changes, add features, fix bugs, etc.
and it is necessary to keep track of your changes and to ensure that your 
code and other artifacts are backed up and protected by being stored on a 
reliable server (or multiple servers) instead of just one machine.  

A \emph{version control system} allows you to ``check-in'' or 
\emph{commit} changes to a code project.  It keeps track of all changes 
and allows you to ``branch'' a code base into a separate copy so that 
you can develop features or enhancements in isolation of the
main code base (often called the ``trunk'' in keeping with the tree 
metaphor).  Once a branch is completed (and well-tested and 
reviewed), it can then be \emph{merged} back into the main trunk 
and it becomes part of the project.

These systems are not only used for organizational and backup 
purposes, but are absolutely essential when developing software 
as part of a team.  Each team member can have their own working 
copy of the project code without interfering with other developer's 
copies or the main trunk.  Only when separate branches have to 
be merged into the trunk do conflicting changes have to be addressed.  
Such a system allows multiple developers to work on a 
very large and complex project in an organized manner.

\begin{figure}[h]
\centering
\includegraphics[scale=.5]{./hack1.0-files/repositorydiagram}
\caption{Trunk, branches, and merging visualization of the Drupal project}
\end{figure}

There are several widely used revision control systems including 
CVS (Concurrent Versions System), SVN (Apache Subversion), and 
Git.  SVN is a \emph{centralized} system: there is a single server that 
acts as the main code repository.  Individual developers can check out 
copies and branch copies (which are also stored in the main repository).  

Git is a \emph{distributed} VCS meaning that multiple servers/computers 
act as full repositories.  Each copy on each developer's machine 
\emph{also} contains a complete revision history.  This makes git a 
decentralized system.  Code commits are committed to a local repository.  
Merging a branch into another requires a push/pull request.  
Decentralizing the system means that anyone's 
machine can act as a code repository and can lead to wider collaboration 
and independence since different parties are no longer dependent on 
one master repository.

Git has become the de facto VCS system in software development.  We have
provided several external resources below, but this Hack will walk you
through the basics of getting started.  You will setup a project with 
git using GitHub (\url{https://github.com}) as your remote server.  You
will then collaborate with someone else to commit changes.

%\part*{On Campus Version}
%
%\section{Installation}
%
%All of the instructions in this Hack will involve using GitHub's 
%Desktop Client.  The screenshots were from the Mac version of this
%program so it may look slightly different on other systems.  There 
%are other alternatives and if you wish to use them you are welcome 
%to do so.
%
%\begin{enumerate}
%  \item Go to \url{https://github.com/} and sign up/register with
%  an account if you have not already done so.
%  \item Download and the client here: \url{https://desktop.github.com/}.
%\end{enumerate}
%
%\section{Creating a Repository}
%
%To focus on the git process, you will create and work with a simple
%``Hello World''-style program but instead of printing ``Hello World'', 
%it will print your name. 
%
%\begin{enumerate}
%  \item Create a new folder (call it \mintinline{text}{hello}) somewhere 
%  on your computer and within it, 
%  create a \mintinline{text}{hello.c} source file with code in it that
%  prints your name.
%  
%  \item Open your GitHub Desktop Client (you may need to sign in with your
%  new GitHub credentials).
%  
%  \item Drag and drop your \mintinline{text}{hello} folder to the GitHub Desktop
%  Client.  It will look something like the following.
%  \begin{center}
%  \includegraphics[scale=0.50]{./hack1.0-files/drag.png}
%  \end{center}
%  Click \mintinline{text}{create a repository} and it should look something
%  like the following.
%  \begin{center}
%  \includegraphics[scale=0.35]{./hack1.0-files/setup.png}
%  \end{center}
%  Provide a simple description and click \mintinline{text}{Create Repository}
%
%  \item If everything is successful, it should look something like the following.
%  \begin{center}
%  \includegraphics[scale=0.125]{./hack1.0-files/postCreation.png}
%  \end{center}
%  The GitHub client will have made an initial commit of the existing
%  files for you.
%
%  \item Push/publish your repository to GitHub.com by clicking 
%  \mintinline{text}{Publish repository} at the top right.  It 
%  should look look something like the following.  
%  \begin{center}
%  \includegraphics[scale=0.35]{./hack1.0-files/publish.png}
%  \end{center}
%  Be sure that \mintinline{text}{Keep this code private} is 
%  \emph{not selected} and click \mintinline{text}{Publish Repository}
%  
%  \item If successful, your repository should now be on GitHub.
%  Point your web browser to \url{https://github.com/login/hello}
%  where \mintinline{text}{login} is replaced with your GitHub user
%  name.  You can browse your repository, view its history, etc.
%  
%\end{enumerate}
%
%\section{Making Changes}
%
%You'll often make changes to you code that you should \emph{commit} to
%your repository.  Though you may make changes and save them to a file, 
%the changes are not saved to the repository's history.  Committing is 
%the action that does this.  Committing only changes your \emph{local}
%repository, the changes will still need to be \emph{pushed} to GitHub.
%In this activity you'll make changes, commit them and then push them
%to GitHub.
%
%\begin{enumerate}
%  \item Open the \mintinline{text}{hello.c} file and add a line that
%  prints your major.  Be sure to save the file.
%
%  \item The GitHub Client should ``see'' these changes.  It should look
%  something like the following. 
%  \begin{center}
%  \includegraphics[scale=0.125]{./hack1.0-files/change.png}
%  \end{center}
%
%  \item Commit your changes by filling out the summary (required) and description.
%  This is your chance to \emph{document} why you made changes to your code.
%  Then click \mintinline{text}{Commit to master}.  Your changes have now
%  been committed to your repository.  
%  \begin{center}
%  \includegraphics[scale=0.125]{./hack1.0-files/commit.png}
%  \end{center}
%  
%  \item Push your changes to GitHub by clicking \mintinline{text}{Push origin}
%  at the top right.  You can verify that your changes have been published
%  to GitHub by refreshing your web browser.
%  \begin{center}
%  \includegraphics[scale=0.125]{./hack1.0-files/pushChange.png}
%  \end{center}
%
%\end{enumerate}
%
%\section{Collaborating With a Team}
%
%In this exercise, you'll need to team up with at least one other person.
%You'll make them a collaborator on your project so they can make changes
%and commit/push them to \emph{your} repository on GitHub.  Alternatively
%you can have them make a pull request, but these instructions do not cover 
%that; refer to one of the resources below for how how to make push/pull
%requests.
%
%\begin{enumerate}
%  \item On GitHub.com, click \mintinline{text}{Settings} in your project.
%  \item In the left menu, click \mintinline{text}{Collaborators}
%  \item Type in your partner's GitHub user name and click \mintinline{text}{Add collaborator}
%\end{enumerate}
%
%\subsection{What your collaborator needs to do}
%
%Together with your partner, walk through the following steps.  These
%steps should be done on \emph{their} computer.
%
%\begin{enumerate}
%  \item Once you've sent an invite to collaborate, they need to accept it.
%  \item Have them navigate their web browser to your GitHub 
%  repository and click \mintinline{text}{Clone or download} and select
%  \mintinline{text}{Open in Desktop}.  This will launch their GitHub
%  Desktop Client and check out \emph{your} code.
%  \item Have them add 2 lines of code to print their name and their major.
%  \item In the GitHub Client, follow the same procedure to commit and push
%  their changes to your repository using the same procedure as above.
%  \item Verify their changes by refreshing your repository.  You can click
%  on the \mintinline{text}{hello.c} and if you both did everything correctly, 
%  you'll see multiple commits by multiple people:
%  \begin{center}
%  \includegraphics[scale=0.25]{./hack1.0-files/completedRepo02.png}
%  \end{center}
%  If you click on \mintinline{text}{History} you can see the changes for
%  each commit: 
%  \begin{center}
%  \includegraphics[scale=0.25]{./hack1.0-files/completedRepo03.png}
%  \end{center}
%
%\end{enumerate}
%
%You are \emph{highly encouraged} to start using git/GitHub (or 
%something similar) for all of your future assignments 
%but be sure to commit code to a \emph{private} repository 
%so that you do not violate the department's academic integrity 
%policy.  
%
%\subsection{Finishing Up}
%
%\begin{enumerate}
%  \item Put \emph{your} GitHub URL into a plain text file named 
%  \mintinline{text}{readme.md}.  Turn this file
%  in using webhandin.  Each individual student will need
%  to hand in their own copy and will receive their own individual
%  grade.
%  \item Verify what you handed in by running the webgrader which will
%  display the contents of your file.
%\end{enumerate}
%
%\part*{Online Version}
%\setcounter{section}{0}

This version of the hack assumes that you are in the online
section or have chosen to use the CS50 IDE.  The following
instructions can also be used for \emph{any} command line 
version of git.

\section{Installation}

Git is already installed on the CS50 IDE and so no further
steps are necessary (though you will need your GitHub account).
However, we do need to do some configuration before we start.

Run the following two commands in your CS50 IDE.  

\mintinline{text}{git config --global user.email "youremail@huskers.unl.edu"}

\mintinline{text}{git config --global user.name "Your Name"}

Where the email and name are substituted with your own.  We 
recommend you use the same email as you used to sign up on GitHub.

\section{Creating a Repository}

To focus on the git process, you will create and work with a simple
``Hello World''-style program but instead of printing ``Hello World'', 
it will print your name. 

\begin{enumerate}
  \item Create a new folder (call it \mintinline{text}{hello}) 
  at the root of your File Navigator and within it, create a 
  \mintinline{text}{hello.c} source file with code in it that
  prints your name.
  
  \item Make this folder (and all of its contents) into a git
  repository by \emph{initializing} it.  Before you do, make sure
  your terminal is in the correct directory (the \mintinline{text}{hello}
  directory).  Recall the following commands from the previous lab:
  \begin{itemize}
    \item \mintinline{text}{pwd} - lists what directory you are currently in
    \item \mintinline{text}{cd hello} - changes your current directory to 
    \mintinline{text}{hello}
    \item \mintinline{text}{ls} - lists the files in the current directory
  \end{itemize}
  
  In the console, change directories to this newly created 
  \mintinline{text}{hello}
  directory and execute the following command:
  
  \mintinline{text}{git init}
  
  It will look something like the following.
  \begin{center}
  \includegraphics[scale=0.50]{./hack1.0-files/cl-gitinit.png}
  \end{center}

  \item This creates a repository on your CS50 IDE server, but it
  does \emph{not} create a repository on GitHub.  We need to do 
  this separately.  Go to your GitHub page (\url{https://github.com/login}
  where \mintinline{text}{login} is replaced with your GitHub
  login) and in the upper right, select \mintinline{text}{New repository}.

  \begin{center}
  \includegraphics[scale=0.50]{./hack1.0-files/cl-githubNewRepo.png}
  \end{center}

  \item Name your repo \mintinline{text}{hello}
  \begin{center}
  \includegraphics[scale=0.50]{./hack1.0-files/cl-githubNewRepoName.png}
  \end{center}

  \item Go back to your CS50 IDE
  
  \item We now need to make our first \emph{commit} which will commit
  our code/changes to the repository.  You can make as many edits as
  you want to your source code files and save them, but you only ``save''
  them to your git repo when you \emph{commit} these changes.  Enter
  the following command.
  
  \mintinline{text}{git add --all}
  
  Which will tell git to add all files in your directory 
  (and subdirectories) to the ``index'' so that changes and
  updates can be ``staged'' for the next commit.  Don't 
  worry about the jargon for now, essentially this just tells
  git to add files to its consideration.
  
  \item Enter the following command.

  \mintinline{text}{git commit -m 'initial commit'}
  
  This commits your changes to the \emph{local} repository
  (the repository on the CS50 IDE server).  It should look something
  like the following.
  
  \begin{center}
  \includegraphics[scale=0.50]{./hack1.0-files/cl-gitcommit}
  \end{center}

  \item We will now \emph{push} these changes to your \emph{remote}
  GitHub repository.  First, we need to specify your GitHub 
  repository as the remote.  Execute the following command:
  
  \mintinline{text}{git remote add origin https://github.com/login/hello.git}
  
  where \mintinline{text}{login} is replaced with your GitHub login.\footnote{
  If you screw this step up and don't replace \mintinline{text}{login} properly, 
  here's how you fix it: 
  \begin{itemize}
    \item Read and follow all instructions going forward
    \item You can use \mintinline{text}{git remote -v} to list the remote URL, 
    verify that you screwed it up
    \item Fix it by running 
    \mintinline{text}{git remote set-url origin https://github.com/login/hello.git}
    \emph{change the login to your login}.
  \end{itemize}
  }
  
  \item Now we can push your commit(s) to the remote repository
  with the following command:
  
  \mintinline{text}{git push -u origin master}
  
  It will likely prompt you for your GitHub username and password.

  \textbf{Update}: as of August 2021, GitHub has changed its security
  requirements.  Instead of providing a password, it now requires you
  to generate a personal access token.  Instructions for how to do this
  are available here.  Be sure to follow these instructions carefully,
  you need to click ``repo'' (Full control of private repositories).
  
  \url{https://docs.github.com/en/github/authenticating-to-github/keeping-your-account-and-data-secure/creating-a-personal-access-token}
  
  Enter your GitHub username and your personal access token as your
  password.  If successful, it should look something like
  the following.\footnote{Note, you \emph{can} configure git 
  to save your username and password; see \url{https://stackoverflow.com/questions/35942754/how-to-save-username-and-password-in-git-gitextension} for
  instructions.  However, since CS50 IDE is a remote server you'd
  be saving your password in plaintext (very bad idea).  That same
  link has instructions for using an SSH Key instead which is
  recommended.}
  
  \begin{center}
  \includegraphics[scale=0.50]{./hack1.0-files/cl-gitPush}
  \end{center}

  \item Your repository should now be on GitHub.
  Point your web browser to \url{https://github.com/login/hello}
  where \mintinline{text}{login} is replaced with your GitHub user
  name.  You can browse your repository, view its history, etc.
  
\end{enumerate}

\subsection{Git Ignore}

Often times there will be files or \emph{artifacts} that you want 
to create, save and work with in your file system \emph{but} you 
don't want them committed to the repository.  For example, when
you compile your \mintinline{text}{hello.c} program, you probably
don't want the executable file, \mintinline{text}{a.out} to be
committed to the repo as you can always rebuild it.  The executable
file is not part of your source code, but an \emph{artifact} of your 
code.  In general, we want git to \emph{ignore} these artifacts.

To do this, we can create a \mintinline{text}{.gitignore} file.  This
is simply a plain text file that contains file and directory names 
that git will ignore.  

\begin{enumerate}
  \item In your CS50 IDE in the \mintinline{text}{hello} directory
  create a new file named \mintinline{text}{.gitignore}.  

  \item Edit it and add \mintinline{text}{a.out} on a single line.
  Save this file.
  
  \item Commit and push this new file to your local and remote
  repositories by executing the series of commands:
  
  \begin{minted}{text}
  git add .gitignore
  git commit -m 'added gitignore file'
  git push -u origin master
  \end{minted}
  
  It should look something like the following.
  
  \begin{center}
  \includegraphics[scale=0.50]{./hack1.0-files/cl-gitignore}
  \end{center}
  
\end{enumerate}

There are many standard \mintinline{text}{.gitignore} files for
various types of projects that you may find useful: 
\url{https://github.com/github/gitignore}

\section{Making Changes}

You'll often make changes to you code that you should periodically
commit to your repository.  Though you may make changes and save them to a file, 
the changes are not saved to the repository's history.  Committing is 
the action that does this.  Committing only changes your \emph{local}
repository, the changes will still need to be \emph{pushed} to GitHub.
In this activity you'll make changes, commit them and then push them
to GitHub.

\begin{enumerate}
  \item Open the \mintinline{text}{hello.c} file and add a line that
  prints your major.  Also, change the line that prints your name
  (add an exclamation point at the end or something).  Be sure to
  save your file and compile and run your program to be sure it is 
  correct.

  \item Git can automatically track these changes, to display the
  changes in your console execute the following command.
  
  \mintinline{text}{git status}
  
  Git will display all the changed files that are not yet \emph{staged}
  (not yet added to the index to be committed).  It should look
  something like the following.
  
  \begin{center}
  \includegraphics[scale=0.5]{./hack1.0-files/cl-gitstatus}
  \end{center}
  
  Note that the \mintinline{text}{a.out} file is not listed (git
  is ignoring it as we directed it to!).
  
  \item Another useful tool allows you to view the changes or
  \emph{differences} to a file.  Execute the following command.
  
  \mintinline{text}{git diff hello.c}

  It should look something like the following.  Deletions (or edits)
  are displayed in red and additions in green.

  \begin{center}
  \includegraphics[scale=0.5]{./hack1.0-files/cl-gitdiff}
  \end{center}
  
  \item We can now commit and push our changes in the same way
  as our initial commit.  Execute the following series of commands.
  
  \begin{minted}{text}
  git add hello.c
  git commit -m 'added my major to the output'
  git push -u origin master  
  \end{minted}

\end{enumerate}
  
Some observations/notes:

\begin{itemize} 
  \item In our last commit, we did not use \mintinline{text}{git add --all} 
  which would have added all (non-ignored) files.  Adding everything at once
  can be convenient but in general commits should only include \emph{related}
  changes and should be as \emph{fine grained} as possible.  You should
  not use commits as a catch-all/save-all operation.

  \item You should \emph{always} provide a good, descriptive commit message
  that accurately reflects your changes.  Commit messages provide good
  documentation on your changes.  For example, when fixing a bug, the commit
  message should reference the original issue or bug report.  

\end{itemize}

\section{Collaborating With a Team}

In this exercise, you'll need to team up with at least one other person.
You'll make them a collaborator on your project so they can make changes
and commit/push them to \emph{your} repository on GitHub.  Alternatively
you can have them make a \emph{pull request}, but these instructions do 
not cover that; refer to one of the resources in the 
\hyperref[section:additionalResources]{Additional Resources} section
for how how to make push/pull requests.

\begin{enumerate}
  \item On the GitHub webpage, click \mintinline{text}{Settings} in your project.
  \item In the left menu, click \mintinline{text}{Manage access}
  \item Click \mintinline{text}{Invite a collaborator} and type 
  in your partner's GitHub user name and click \mintinline{text}{Add}
\end{enumerate}

\subsection{What your collaborator needs to do}

Together with your partner, walk through the following steps.  These
steps should be done on \emph{their} computer.

\begin{enumerate}
  \item Once you've sent an invite to collaborate, they need to 
  accept it.
  \item Your partner should create a new directory at the root of
  their CS50 IDE (or whatever they are using) to hold \emph{your}
  repository.
  \item Your partner should execute the following command inside
  of the new directory:
  
  \mintinline{text}{git clone https://github.com/login/hello}
  
  Where again \mintinline{text}{login} is replaced with your
  GitHub login.  This clones the \emph{remote} repository and now you have a 
  \emph{local} version of it!

  \item Your partner should add 2 lines of code to 
  print their name and their major.
  
  \item Your partner should follow the same procedure to commit 
  and push their changes to \emph{your remote} repository using 
  the same procedure as they did with theirs.
  
  \item Verify their changes by refreshing your repository on
  GitHub.  You can click on the \mintinline{text}{hello.c} and if 
  you both did everything correctly, you'll see multiple commits 
  by multiple people:
  \begin{center}
  \includegraphics[scale=0.25]{./hack1.0-files/completedRepo02.png}
  \end{center}
  If you click on \mintinline{text}{History} you can see the changes for
  each commit: 
  \begin{center}
  \includegraphics[scale=0.25]{./hack1.0-files/completedRepo03.png}
  \end{center}

\end{enumerate}

Now, go back to \emph{your} CS50 IDE.  Remember, your partner's
changes were \emph{pushed} to your \emph{remote} repository hosted
on GitHub.  If you look at your \mintinline{text}{hello.c} file
you won't see their changes because this is your \emph{local}
repository.  

In order to get your partner's changes you'll need to \emph{pull}
their changes from your remote repository to your local repository.
To do this, run the following command:

\mintinline{text}{git pull}

As long as there are no conflicting changes, all of your partner's
changes should now be reflected in your local repository.  Of course,
you'll need to repeat this process for your partner's repository (add
your name/major to their and commit it so they can pull your changes).  

You are \emph{highly encouraged} to start using git/GitHub (or 
something similar) for all of your future assignments 
but be sure to commit code to a \emph{private} repository 
so that you do not violate the department's academic integrity 
policy.  

\subsection{Finishing Up}

\begin{enumerate}
  \item Put the URL for \emph{your} GitHub repository into a plain text file named 
  \mintinline{text}{readme.md}.  Turn this file
  in using webhandin.  Each individual student will need
  to hand in their own copy and will receive their own individual
  grade.  Your file should look \emph{something} like this:
  \begin{minted}{text}
   https://github.com/cbourke/hello
  \end{minted}
  \item Verify what you handed in by running the webgrader which will
  display the contents of your file.
  \item \textbf{Be sure your repo is public}
\end{enumerate}

\section*{Additional Instructions}

\begin{itemize}
  \item You are encouraged to collaborate with any number of students 
  before, during, and after your scheduled hack session.    
  \item Each student is responsible for \emph{their} repository, so 
  if/when you team up with a partner, you'll need to go through this
  Hack at least twice: once as the primary repository owner and once as
  a collaborator.
\end{itemize}
  
\section*{Additional Resources}
\label{section:additionalResources}

\begin{itemize} 
  \item Video tutorial on Github Desktop: \url{https://www.youtube.com/watch?v=kFix7UDJ7LA}
  \item Interactive git tutorial: \url{https://try.github.io/levels/1/challenges/1}
  \item Pro Git, free online book: \url{https://git-scm.com/book/en/v2}
\end{itemize}


  


\end{document}
