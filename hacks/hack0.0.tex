\documentclass[12pt]{scrartcl}

\input{preamble}

\makeatletter
\title{Hack 0.0 (template)}\let\Title\@title
\subtitle{Computer Science I\\
{\small
\vskip1cm
Department of Computer Science \& Engineering \\
University of Nebraska--Lincoln}
\vskip-1cm}
%\author{Dr.\ Chris Bourke}
\date{~}
\makeatother

\begin{document}

\maketitle

\hrule

\section*{Introduction}

Hack session activities are small weekly programming assignments intended
to get you started on full programming assignments.  Collaboration is allowed
and, in fact, \emph{highly encouraged}.  You may start on the activity before
your hack session, but during the hack session you must either be actively 
working on this activity or \emph{helping others} work on the activity.
You are graded using the same rubric as assignments so documentation, style, 
design and correctness are all important.  This activity is \textbf{due 
at 23:59:59 on the Monday} in the week in which it is assigned according 
to the CSE system clock.

%TODO: rubric 
% instructions
% documentation
% style
% design including input validation
% correctness

\section*{Problem Statement}

\mintinline{text}{input} 

your output should look something like the following:

\begin{minted}{text}
output
\end{minted}

\section*{Instructions}

\begin{itemize}
  \item You are encouraged to collaborate any number of students 
  before, during, and after your scheduled hack session.  
  \item Design at least 3 test cases \emph{before} you begin
  designing or implementing your program.  Test cases are 
  input-output pairs that are known to be correct using means
  other than your program.
  \item Include the name(s) of everyone who worked together on
  this activity in your source file's header.
  \item Name your program \mintinline{text}{TODO.c}, and
  turn it in via webhandin, making sure that it runs and executes
  correctly in the webgrader.  Each individual student will need
  to hand in their own copy and will receive their own individual
  grade.
\end{itemize}
  


\end{document}
