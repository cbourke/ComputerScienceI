\documentclass[12pt]{scrartcl}


\usepackage{epsfig,amssymb}

\usepackage{xcolor}
\usepackage{graphicx}
\usepackage{epstopdf}
\usepackage{multirow}
\usepackage{float}

\definecolor{darkred}{rgb}{0.5,0,0}
\definecolor{darkgreen}{rgb}{0,0.5,0}
\usepackage[pdfusetitle]{hyperref}
\hypersetup{
  letterpaper,
  colorlinks,
  linkcolor=red,
  citecolor=darkgreen,
  menucolor=darkred,
  urlcolor=blue,
  pdfpagemode=none,
}

\usepackage{fullpage}
\usepackage{tikz}
\pagestyle{empty} %
%obsolete: \usepackage{subfigure}
%use: 
\usepackage{subcaption}

\definecolor{MyDarkBlue}{rgb}{0,0.08,0.45}
\definecolor{MyDarkRed}{rgb}{0.45,0.08,0}
\definecolor{MyDarkGreen}{rgb}{0.08,0.45,0.08}

\definecolor{mintedBackground}{rgb}{0.95,0.95,0.95}
\definecolor{mintedInlineBackground}{rgb}{.90,.90,1}

\usepackage[newfloat=true]{minted}

\setminted{mathescape,
           linenos,
           autogobble,
           frame=none,
           framesep=2mm,
           framerule=0.4pt,
           %label=foo,
           xleftmargin=2em,
           xrightmargin=0em,
           %startinline=true,  %PHP only, allow it to omit the PHP Tags *** with this option, variables using dollar sign in comments are treated as latex math
           numbersep=10pt, %gap between line numbers and start of line
           style=default} %syntax highlighting style, default is "default"

\setmintedinline{bgcolor={mintedBackground}}
%doesn't work with the above workaround:
\setminted{bgcolor={mintedBackground}}
\setminted[text]{bgcolor={mintedBackground},linenos=false,autogobble,xleftmargin=1em}
%\setminted[php]{bgcolor=mintedBackgroundPHP} %startinline=True}
\SetupFloatingEnvironment{listing}{name=Code Sample}
\SetupFloatingEnvironment{listing}{listname=List of Code Samples}

\setlength{\parindent}{0pt} %
\setlength{\parskip}{.25cm}
\newcommand{\comment}[1]{}

\usepackage{amsmath}
\usepackage{algorithm2e}
\SetKwInOut{Input}{input}
\SetKwInOut{Output}{output}
%NOTE: you can embed algorithms in solutions, but they cannot be floating objects; use [H] to make them non-floats

\usepackage{lastpage}

%\usepackage{titling}
\usepackage{fancyhdr}
\renewcommand*{\titlepagestyle}{fancy}
\pagestyle{fancy}
%\fancyhf{}
%\rhead{Computer Science I}
%\lhead{Guides and tutorials}
\renewcommand{\headrulewidth}{0.0pt}
\renewcommand{\footrulewidth}{0.4pt}
\lfoot{\Title\ -- Computer Science I}
\cfoot{~}
\rfoot{\thepage\ / \pageref*{LastPage}}


\makeatletter
\title{Hack 12.0}\let\Title\@title
\subtitle{Computer Science I\\
Recursion \& Memoization\\
{\small
\vskip1cm
Department of Computer Science \& Engineering \\
University of Nebraska--Lincoln}
\vskip-3cm}
%\author{Dr.\ Chris Bourke}
\date{~}
\makeatother

\begin{document}

\maketitle

\hrule

\section*{Introduction}

Hack session activities are small weekly programming assignments intended
to get you started on full programming assignments.  Collaboration is allowed
and, in fact, \emph{highly encouraged}.  You may start on the activity before
your hack session, but during the hack session you must either be actively 
working on this activity or \emph{helping others} work on the activity.
You are graded using the same rubric as assignments so documentation, style, 
design and correctness are all important.  This activity is due at 23:59:59
on the Monday following the hack session in which it is assigned according
to the CSE system clock.

%TODO: rubric 
% instructions
% documentation
% style
% design including input validation
% correctness



Correctness: proportionally for test cases; it \textbf{must run on the grader} to
receive any points.


\section*{Problem Statement}

A binomial coefficient, ``$n$ choose $k$'' is a number that corresponds 
to the number of ways to \emph{choose} $k$ items from a set of $n$ distinct
items.  You may be familiar with some the notations, $C(n,k)$ or $C_n^k$ 
or ${}_{n}C_k $, but most commonly this is written as 
  $${n \choose k}$$
and read as ``$n$ choose $k$''.  There is an easy to compute formula involving
factorials:
  $${n \choose k} = \frac{n!}{(n-k)!k!}$$
For example, if we have $n = 4$ items, say $\{a, b, c, d\}$ and want to choose
$k=2$ of them, then there are 
  $${4 \choose 2} = \frac{4!}{(4-2)!2!} = 6$$
ways of doing this.  The six ways are:
  $$\{a, b\}, \{a, c\}, \{a, d\}, \{b, c\}, \{b, d\}, \{c, d\}$$
There are a lot of other interpretations and applications for binomial 
coefficients, but this hack will focus on computing their value using
a different formula, Pascal's Rule\footnote{Which can be used to generate
Pascal's Triangle, \url{https://en.wikipedia.org/wiki/Pascals_triangle}}:
  $${n \choose k} = {n-1 \choose k} + {n-1 \choose k-1}$$
which is a recursive formula.  The base cases for Pascal's Rule are when
$k = 0$ and $n = k$.  In both cases, the value is 1.  When $k = 0$, we are
not choosing any elements and so there is only one way of doing that (i.e.\
choose nothing).  When $n = k$ we are choosing every element, again there
is only one way of doing that.  

\subsection*{Writing a Naive Recursion}

Implement and test the following function \emph{using a recursive} 
solution:

\mintinline{c}{long choose(int n, int k);}

which takes $n$ and $k$ and computes ${n\choose k}$ using Pascal's Rule.
Note that the return type is a \mintinline{c}{long}\footnote{For those
using Windows, you may need to instead use a \mintinline{c}{long long}
data type to get a 64-bit integer.} which is a 64-bit
integer allowing you to compute values up to 
  $$2^{63}-1 = 9,223,372,036,854,775,807$$
(a little over 9 quintillion).  Write a \mintinline{c}{main} function
that takes $n$ and $k$ as command line arguments and outputs the result
to the standard output so you can easily test it.

\subsection*{Benchmarking}

Run your program on values of $n, k$ in Table \ref{table:easyValues} 
and time (roughly) how long it takes your program to execute.  You
can check your solutions with an online tool such as 
\url{https://www.wolframalpha.com/}.

\begin{table}[ht]
\centering
\begin{tabular}{c|c}
$n$ & $k$ \\
\hline\hline
4 & 2 \\
10 & 5 \\
32 & 16 \\  %5 seconds
34 & 17 \\  %15 seconds
36 & 18 \\ %60 seconds
\end{tabular}
\caption{Test Values}
\label{table:easyValues}
\end{table}

Now formulate an estimate of how long your program would take to 
execute with larger values.  You can make a \emph{rough} estimate 
how many function calls are made using the binomial value itself.  
That is, to compute ${n \choose k}$ using Pascal's Rule would make 
\emph{about} ${n \choose k}$ function calls.

Use the running time of your program from the test values to 
estimate how long your program would run for the values in 
Table \ref{table:hardValues}.

\begin{table}[ht]
\centering
\begin{tabular}{c|r}
${n \choose k}$ &  value \\
\hline\hline
${54 \choose 27}$ & =     1,946,939,425,648,112 \\
${56 \choose 28}$ & =     7,648,690,600,760,440 \\
${58 \choose 29}$ & =    30,067,266,499,541,040 \\
${60 \choose 30}$ & =   118,264,581,564,861,424 \\
${62 \choose 31}$ & =   465,428,353,255,261,088 \\
${64 \choose 32}$ & = 1,832,624,140,942,590,534 \\
${66 \choose 33}$ & = 7,219,428,434,016,265,740 \\
\end{tabular}
\caption{Larger Values}
\label{table:hardValues}
\end{table}

\subsection*{Improving Performance with Memoization}

You'll now improve your program's performance using memoization
to avoid unnecessary repeated recursive calls.  

\begin{enumerate}
  \item Write code (either in the \mintinline{c}{main} function or 
  using another ``entry point'' function) to create a memoization table 
  containing \mintinline{c}{long} values of dimension $(n+1) \times (k+1)$
  \item Initialize the values in the table to $-1$ as a flag value
  to indicate that the value in the table has not yet been set.
  \item Using your previous recursive implementation as a guide, write
  a new recursive function that also takes the table as a parameter.
  When the function needs to compute ${n \choose k}$ it checks the table
  first: if the value has already been computed (is not $-1$) then it
  returns that value.  Otherwise, it performs the recursive computation.
  Before returning the value, however, it should store it (\emph{cache}
  it) in the table so that subsequent computations avoid the recursion.  
  \item Modify your \mintinline{c}{main} function to use this more
  efficient version and re-test it with the values above.  Compare the
  time it took using memoization versus the naive recursion.
  \item Rerun your program with the values in Tables \ref{table:easyValues} 
  and \ref{table:hardValues} to verify they work and note the difference
  in running time.  
\end{enumerate}

\section*{Instructions}

\begin{itemize}

  \item Place all of your function definitions in a source file named 
  \mintinline{text}{binomial.c} and hand it in with your header file, 
  \mintinline{text}{binomial.h}.  Place your \mintinline{c}{main} 
  function in a file named \mintinline{text}{binomialDemo.c} 

  \item You are encouraged to collaborate any number of students 
  before, during, and after your scheduled hack session.  

  \item Include the name(s) of everyone who worked together on
  this activity in your source file's header.

  \item Turn in all of your files via webhandin, making sure that 
  it runs and executes correctly in the webgrader.  Each individual 
  student will need to hand in their own copy and will receive 
  their own individual grade.
\end{itemize}  


\end{document}
